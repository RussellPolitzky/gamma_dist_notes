% Options for packages loaded elsewhere
% Options for packages loaded elsewhere
\PassOptionsToPackage{unicode}{hyperref}
\PassOptionsToPackage{hyphens}{url}
\PassOptionsToPackage{dvipsnames,svgnames,x11names}{xcolor}
%
\documentclass[
  letterpaper,
  DIV=11,
  numbers=noendperiod]{scrartcl}
\usepackage{xcolor}
\usepackage{amsmath,amssymb}
\setcounter{secnumdepth}{-\maxdimen} % remove section numbering
\usepackage{iftex}
\ifPDFTeX
  \usepackage[T1]{fontenc}
  \usepackage[utf8]{inputenc}
  \usepackage{textcomp} % provide euro and other symbols
\else % if luatex or xetex
  \usepackage{unicode-math} % this also loads fontspec
  \defaultfontfeatures{Scale=MatchLowercase}
  \defaultfontfeatures[\rmfamily]{Ligatures=TeX,Scale=1}
\fi
\usepackage{lmodern}
\ifPDFTeX\else
  % xetex/luatex font selection
\fi
% Use upquote if available, for straight quotes in verbatim environments
\IfFileExists{upquote.sty}{\usepackage{upquote}}{}
\IfFileExists{microtype.sty}{% use microtype if available
  \usepackage[]{microtype}
  \UseMicrotypeSet[protrusion]{basicmath} % disable protrusion for tt fonts
}{}
\makeatletter
\@ifundefined{KOMAClassName}{% if non-KOMA class
  \IfFileExists{parskip.sty}{%
    \usepackage{parskip}
  }{% else
    \setlength{\parindent}{0pt}
    \setlength{\parskip}{6pt plus 2pt minus 1pt}}
}{% if KOMA class
  \KOMAoptions{parskip=half}}
\makeatother
% Make \paragraph and \subparagraph free-standing
\makeatletter
\ifx\paragraph\undefined\else
  \let\oldparagraph\paragraph
  \renewcommand{\paragraph}{
    \@ifstar
      \xxxParagraphStar
      \xxxParagraphNoStar
  }
  \newcommand{\xxxParagraphStar}[1]{\oldparagraph*{#1}\mbox{}}
  \newcommand{\xxxParagraphNoStar}[1]{\oldparagraph{#1}\mbox{}}
\fi
\ifx\subparagraph\undefined\else
  \let\oldsubparagraph\subparagraph
  \renewcommand{\subparagraph}{
    \@ifstar
      \xxxSubParagraphStar
      \xxxSubParagraphNoStar
  }
  \newcommand{\xxxSubParagraphStar}[1]{\oldsubparagraph*{#1}\mbox{}}
  \newcommand{\xxxSubParagraphNoStar}[1]{\oldsubparagraph{#1}\mbox{}}
\fi
\makeatother


\usepackage{longtable,booktabs,array}
\usepackage{calc} % for calculating minipage widths
% Correct order of tables after \paragraph or \subparagraph
\usepackage{etoolbox}
\makeatletter
\patchcmd\longtable{\par}{\if@noskipsec\mbox{}\fi\par}{}{}
\makeatother
% Allow footnotes in longtable head/foot
\IfFileExists{footnotehyper.sty}{\usepackage{footnotehyper}}{\usepackage{footnote}}
\makesavenoteenv{longtable}
\usepackage{graphicx}
\makeatletter
\newsavebox\pandoc@box
\newcommand*\pandocbounded[1]{% scales image to fit in text height/width
  \sbox\pandoc@box{#1}%
  \Gscale@div\@tempa{\textheight}{\dimexpr\ht\pandoc@box+\dp\pandoc@box\relax}%
  \Gscale@div\@tempb{\linewidth}{\wd\pandoc@box}%
  \ifdim\@tempb\p@<\@tempa\p@\let\@tempa\@tempb\fi% select the smaller of both
  \ifdim\@tempa\p@<\p@\scalebox{\@tempa}{\usebox\pandoc@box}%
  \else\usebox{\pandoc@box}%
  \fi%
}
% Set default figure placement to htbp
\def\fps@figure{htbp}
\makeatother





\setlength{\emergencystretch}{3em} % prevent overfull lines

\providecommand{\tightlist}{%
  \setlength{\itemsep}{0pt}\setlength{\parskip}{0pt}}



 


\KOMAoption{captions}{tableheading}
\makeatletter
\@ifpackageloaded{caption}{}{\usepackage{caption}}
\AtBeginDocument{%
\ifdefined\contentsname
  \renewcommand*\contentsname{Table of contents}
\else
  \newcommand\contentsname{Table of contents}
\fi
\ifdefined\listfigurename
  \renewcommand*\listfigurename{List of Figures}
\else
  \newcommand\listfigurename{List of Figures}
\fi
\ifdefined\listtablename
  \renewcommand*\listtablename{List of Tables}
\else
  \newcommand\listtablename{List of Tables}
\fi
\ifdefined\figurename
  \renewcommand*\figurename{Figure}
\else
  \newcommand\figurename{Figure}
\fi
\ifdefined\tablename
  \renewcommand*\tablename{Table}
\else
  \newcommand\tablename{Table}
\fi
}
\@ifpackageloaded{float}{}{\usepackage{float}}
\floatstyle{ruled}
\@ifundefined{c@chapter}{\newfloat{codelisting}{h}{lop}}{\newfloat{codelisting}{h}{lop}[chapter]}
\floatname{codelisting}{Listing}
\newcommand*\listoflistings{\listof{codelisting}{List of Listings}}
\makeatother
\makeatletter
\makeatother
\makeatletter
\@ifpackageloaded{caption}{}{\usepackage{caption}}
\@ifpackageloaded{subcaption}{}{\usepackage{subcaption}}
\makeatother
\usepackage{bookmark}
\IfFileExists{xurl.sty}{\usepackage{xurl}}{} % add URL line breaks if available
\urlstyle{same}
\hypersetup{
  pdftitle={Equivalence of Gamma GLM and Volume-Weighted Chain Ladder},
  pdfauthor={Actuarial Proof},
  colorlinks=true,
  linkcolor={blue},
  filecolor={Maroon},
  citecolor={Blue},
  urlcolor={Blue},
  pdfcreator={LaTeX via pandoc}}


\title{Equivalence of Gamma GLM and Volume-Weighted Chain Ladder}
\author{Actuarial Proof}
\date{}
\begin{document}
\maketitle


\subsection{1. Chain Ladder Development
Factors}\label{chain-ladder-development-factors}

Let \(C_{k,j}\) denote the cumulative claims amount for origin period
\(k = 1,\dots,K\) and development period \(j = 0,\dots,J\), where
\(C_{k,0}\) represents the initial cumulative claims.

For each development period \(j \ge 1\), define the individual
development ratio:

{[} Y\_\{k,j\} = \frac{C_{k,j}}{C_{k,j-1}}, \qquad k = 1,\dots,K-j+1,
\tag{1} {]}

where the upper limit on \(k\) ensures both \(C_{k,j-1}\) and
\(C_{k,j}\) are observed.

The \textbf{volume-weighted chain ladder development factor} for period
\(j\) is:

{[} f\_j =
\frac{\sum_{k=1}^{K-j+1} C_{k,j}}{\sum_{k=1}^{K-j+1} C_{k,j-1}}. \tag{2}
{]}

\subsection{2. GLM Specification}\label{glm-specification}

We model the response \(Y_{k,j}\) using a Gamma distribution with mean
\(\mu_j\) and constant shape parameter \(\nu\).\\
The mean depends only on the development period \(j\) via an
\textbf{inverse link function}:

{[} g(\mu\_j) = \frac{1}{\mu_j} = \eta\_j, \tag{3} {]}

where the linear predictor \(\eta_j = \beta_j\) is the coefficient for
development period \(j\) (treated as a categorical variable without
intercept).

The model uses \textbf{weights} equal to the claims amount at the
beginning of each development period:

{[} w\_\{k,j\} = C\_\{k,j-1\}. \tag{4} {]}

Observations are assumed independent.

\subsection{3. Log-Likelihood and Estimating
Equations}\label{log-likelihood-and-estimating-equations}

Ignoring constants, the weighted log-likelihood for the Gamma GLM is:

{[} \ell = \sum\emph{\{j=1\}\^{}\{J\} \sum}\{k=1\}\^{}\{K-j+1\}
w\_\{k,j\} \left[ -\nu \frac{Y_{k,j}}{\mu_j} - \nu \log \mu_j \right].
\tag{5} {]}

Since the means \(\mu_j\) are separate for each \(j\), we can consider
the contribution for a fixed \(j\):

{[} \ell\emph{j = -\nu \sum}\{k=1\}\^{}\{K-j+1\} w\_\{k,j\} \left(
\frac{Y_{k,j}}{\mu_j} + \log \mu\_j \right). \tag{6} {]}

Differentiating with respect to \(\mu_j\):

{[} \frac{\partial \ell_j}{\partial \mu_j} =
-\nu \sum\emph{\{k=1\}\^{}\{K-j+1\} w}\{k,j\} \left(
-\frac{Y_{k,j}}{\mu_j^2} + \frac{1}{\mu_j} \right) = \frac{\nu}{\mu_j^2}
\sum\emph{\{k=1\}\^{}\{K-j+1\} w}\{k,j\} \left( Y\_\{k,j\} - \mu\_j
\right). \tag{7} {]}

Setting the derivative to zero gives the maximum likelihood estimating
equation:

{[} \sum\emph{\{k=1\}\^{}\{K-j+1\} w}\{k,j\} \left( Y\_\{k,j\} - \mu\_j
\right) = 0. \tag{8} {]}

\subsection{\texorpdfstring{4. Solution for
\(\mu_j\)}{4. Solution for \textbackslash mu\_j}}\label{solution-for-mu_j}

From equation (8), we obtain:

{[} \mu\_j =
\frac{\sum_{k=1}^{K-j+1} w_{k,j} Y_{k,j}}{\sum_{k=1}^{K-j+1} w_{k,j}}.
\tag{9} {]}

Now substitute the definitions of \(w_{k,j}\) and \(Y_{k,j}\) from
equations (4) and (1):

{[} \sum\emph{\{k=1\}\^{}\{K-j+1\} w}\{k,j\} Y\_\{k,j\} =
\sum\emph{\{k=1\}\^{}\{K-j+1\} C}\{k,j-1\}
\cdot \frac{C_{k,j}}{C_{k,j-1}} = \sum\emph{\{k=1\}\^{}\{K-j+1\}
C}\{k,j\}, \tag{10} {]}

{[} \sum\emph{\{k=1\}\^{}\{K-j+1\} w}\{k,j\} =
\sum\emph{\{k=1\}\^{}\{K-j+1\} C}\{k,j-1\}. \tag{11} {]}

Therefore:

{[} \mu\_j =
\frac{\sum_{k=1}^{K-j+1} C_{k,j}}{\sum_{k=1}^{K-j+1} C_{k,j-1}} = f\_j,
\tag{12} {]}

which is exactly the volume-weighted chain ladder development factor
defined in equation (2).

\subsection{5. GLM Coefficients}\label{glm-coefficients}

Using the inverse link function \(\eta_j = 1/\mu_j\), the estimated GLM
coefficient for development period \(j\) is:

{[} \hat{\beta}\_j = \hat{\eta}\_j = \frac{1}{\hat{\mu}_j} =
\frac{1}{f_j}. \tag{13} {]}

Thus, the GLM coefficients are the \textbf{reciprocals} of the chain
ladder development factors.

\subsection{6. Predicted Values}\label{predicted-values}

For any observation in development period \(j\), the fitted mean is
\(\hat{\mu}_j = f_j\).\\
Hence, for a future cumulative claim given \(C_{k,j-1}\), the model
predicts:

{[} \mathbb{E}{[}C\_\{k,j\} \mid C\_\{k,j-1\}{]} = C\_\{k,j-1\}
\cdot \hat{\mu}\emph{j = C}\{k,j-1\} \cdot f\_j, \tag{14} {]}

which matches the chain ladder projection exactly.

\subsection{7. Conclusion}\label{conclusion}

The Gamma GLM with: - inverse link function, - development period as a
categorical predictor, - weights equal to the claims reserved at the
start of each development period,

yields coefficient estimates \(\hat{\beta}_j = 1/f_j\) and fitted means
\(\hat{\mu}_j = f_j\), where \(f_j\) are the volume-weighted chain
ladder development factors.\\
Consequently, this GLM formulation reproduces the deterministic chain
ladder forecasts exactly.




\end{document}
